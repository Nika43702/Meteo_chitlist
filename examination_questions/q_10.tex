% !TEX root = ../main.tex

\section{Хар-стики влажности}
\textbf{Упругость водяного пара [е]=[гПа, мм.рт.ст]} – парциальное давление водяного пара, измеренное в ед давления\\
\textbf{Упругость насыщения [Е]=[гПа]} – максимально возможная упругость при данной температуре\\
При одной и той же T упругость насыщения больше над водой, меньше надо льдом.\\
\textbf{Точка росы $[ t_{d} $ или $ \uptau ]$} – T, при которой воздух достигает состояния насыщения при данном содержании водяного пара и неизменном давлении.\\
\textbf{Дефицит точки росы $[\Delta t_{d} ]$} – разность T и точки росы.  $[\Delta t_{d}=t-t_{d}]$ \\
\textbf{Абсолютная влажность [a]=[г/м$^3$]} – масса водяного пара в граммах в одном кубическом метре воздуха\\
a=0,8e/(1+$\alpha$ t), $\alpha$ =0,004 – термический коэффициент объемного расширенного воздуха (насколько газ расширится при T↑ на 1\celsius)\\
\textbf{Удельная влажность} (массовая доля водяного пара) \textbf{[q]=[г/кг]} – количество водяного пара в граммах в 1 кг воздуха.\\
q=622e/(P-0,378e)	\\ 
\textbf{Относительная влажность [f]=[\%]} – процентное отношение фактической упругости водяного пара (e) к упругости насыщения (E)\\
F=e/E*100\%=a/A*100\%  A – предельная влажность (max a) при данной температуре\\ 
\textbf{Дефицит насыщения [d]=[гПа]} – разность упругости насыщения и упругости водяного пара\\
d=E-e\\
\textbf{Виртуальная температура $[t_v]$} – расчетная T сухого воздуха, при котором его плотность равна плотности влажного воздуха.\\
tv=t(1+0,608q) [\celsius]\\
Tv=T(1+0,378e/p) [K]\\
$\delta tv-0,608qt$ - виртуальный добавок\\
q - удельная влажность [г/кг]\\
Введение $T_v$ позволяет учитывать влияние влажности на плотность воздуха (чем более влажный, тем менее плотный)\\
Виртуальная темп выше сухого воздуха на значение виртуального добавка.