% !TEX root = ../main.tex

\section{Методы и средства измерения f}
\underline{\textbf{Психрометр}} по разницам в температуре вычисляются: упругость водяного пара, отн влажность, точка росы и др.
\begin{itemize} 
	\item \textbf{Станционный (стационарный) психрометр} (в будке): 2 ртутных термометров: 	
	\begin{enumerate} 
		\item "Сухой" 
		\item Обернут кусочком батиста(ткань), погружен в стакан с водой 
	\end{enumerate} 	
	Разница T между термометровами (1>2) используется для определения влажности по психометрическим таблицам
	\item\textbf{Аспирационный:} Для влажности и t в помещении и в полевых условиях. \\ 
	2 ртутных термометра+аспирационная головка\\
	Воздух через вентилятор всасывается, обтекает резервуар термометров и выбрасывается наружу черещ прорези в аспирационной головке
\end{itemize} 

\underline{\textbf{Гигрометр:}} для наблюдений за отн влажностью (сразу показывает). влажность при ниже (-10)\celsius 
\begin{itemize} 
	\item Волосной: обезжиренный человеческий волос изменяет длину пр разной отн влажности\\
	отн. влажность. Пределы: 30-100\% 
	\item Пленочный: деформация гигроскопической органической пленки\\
	отн. влажность. Пределы: 20-100\% 
\end{itemize} 

\textbf{Гигрограф:} для непрерывной регистрации изм отн влажности  (самописец):
\begin{itemize}
	\item \textbf{Волосной}(не является абсолютным прибором, надо вносить порпавки по психометру)
	\item  \textbf{Пленочный} (измерение размеров органической пленки при изменении влажности через передаточный механизм преобразуется в перемещение стрелки с пером по диаграммной ленте, закрепленной на барабане часового механизма)
\end{itemize}

