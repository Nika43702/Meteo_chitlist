% !TEX root = ../main.tex

\section{Элементы термодинамики атмосферы2}
При подъеме выше происходит конденсация пара и образование облаков. Их граница обычно на 100-200м выше уровня конденсации, т.к. для образования видимого облака лолжно сконденсироваться опр кол-во влаги, для чего требуется добавочное охлаждение насыщ воздуха ниже точки росы\\
\textbf{Кривая состояния} - графическое изображение адаб измен-ия T в подним-ейся воз массе при любых T и P: нижняя часть до уровня конденсации - сухая адиабата; выше - влажная адиабата\\
\par \underline{\textbf{Условия верт устойчивости атмосферы}}\\
Причины начального возн-ия верт-ых движений:
\begin{itemize}
	\item отличие T в некотором воз объеме от $T_\text{окр.возд}$
	\item турбулентность воздуха
	\item натекание воз потоков на неровности зем поверхн.
	\item вынужденный подъем теплого воздуха по хлину холодного на атс фронтах
\end{itemize}
Дальнейшее развитие и интенсивность зависят от соотношения T подним/опуск объема воздуха и T окр воздуха\\
\textbf{Случаи развития верт движений:}
\begin{enumerate}
	\item \textbf{Неустойчивое равновесие воздуха:}\\
	Если верт градиент $T_\text{окр.возд}$ > сухоад-ого и влажноад-ого, то массы возд-ха становятся неустойчивыми (т.е. будут подниматься/опускаться)
	\item \textbf{Устойчивое равновесие}\\
	Если верт градиент $T_\text{окр.возд}$ < влажноад-ого и сухоад-ого, то массы будут возращаться в прежнее устойчивое состояние
	\item \textbf{Влажно-неустойчивое равновесие} (влажнонеустойчивость)\\
	Если верт градиент $T_\text{окр.возд}$ < сухоад-ого, но > влажноад-ого, то только при подъеме насыщ воздуха будет неустойчивое равновесие
	\item \textbf{Безразличное равновесие}\\
	Если верт градиент $T_\text{окр.возд}$= сухоад-ому или влажноад-ому, то некоторый объем воздуха, поднятый/опущенный на какую-л. высоту, здесь и останется
\end{enumerate}
В насыщ воздухе восходящие движения возникают легче, чем в сухом. При одном и том же верт градиенте насыщенный вохдух всегда более неустойчив\\
При неустойчивом состоянии верт движения воздуха интенсивно развиваются. Начавшееся движ-е воздуха вверх/вниз продолжается с возрастающей скоростью.  Полет сопровождается бросками самолета и болтанкой\\
Слои, препятсвующие развитию верт движений - \textbf{инверсии}, т.к. с высотой темп увеличивается, то подним воз масса, дойдя до этого слоя прекращает движение\\
\par \underline{\textbf{Уровень конвекции}} - высота, до которой распространяется восходящий воз поток (при верт движ-ях воз масса поднимается до тех пор, пока T не сравняется с $T_\text{окр.возд}$)\\
Если уровень конвекции выше уровня конденсации, то между слоями возникают обалка\\
Уровень конвекции является верхней границей болтанки саолетов, вызываемой неустойчивым состоянием атмосферы




 
