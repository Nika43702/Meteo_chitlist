% !TEX root = ../main.tex

\section{Осдаки2}
\textbf{Морось} - жидкие осадки в виде мелких капель, d<0,5мм; парят в воздухе. Сухая поверхность намакает медленно и равномерно; на воде следов нет\\
\textbf{Морось Переохлажденная} - морось при t=-15-0t$\celsius$; образует гололед\\

\textbf{Снежные зерна} - твердые осадки в виде мелких белых непрозрачных частиц при -t$\celsius$; d<2мм\\

\textbf{Ливневый(ливневой) снег} - снег ливневого хар-ка с резкими колебаниями гор. видимости\\
\textbf{Ливневый дождь со снегом} - смешанные осадки ливневого хар-ка при -t$\celsius$ в виде смеси капель и снежинок. Образуется гололед из-за намерзания на предметы\\

\textbf{Снежная крупа} - твердые осадки ливневого хар-ра при 0$\celsius$; похожи на белые непрозрачные крупинки; d=2-5мм; часто выпадает с ливневым снегом или перед ним. Легко раздавливаются пальцами\\
\textbf{Ледяная крупа} - твердые осадки ливневого хар-ка в виде прозрачных или полупрозрачных крупинок льда при t=-5-(+10)t$\celsius$; d=1-3мм. В центре курпинок непрозрачное ядро. Раздавливать пальцами сложно; отскакивают от поверхности.\\
Редко покрыты водяной пленкой  или выпадают с каплями. При 0t$\celsius$ крупинки смерзаются на поверхности, образуя гололед\\

\textbf{Град} - твердые осадки, в теплый период в виде кусочков льда разных размеров и формы (чаще d=2-5мм) при t от +10t$\celsius$ (чаще всего). Выпадает от 1-2 до 10-20 минут. Часто сопровождается грозой и проливным дождем\\

\underline{\textbf{Неклассифицированные осадки:}}\\
\textbf{Ледяные иглы} - твердые осадки в виде мелких кристаллов изо льда, паращих в воздухе. Образуются в морозную погоду при t=-10..-15t$\celsius$. Сверкают в солн/лунных лучах или при свете фонаря. Часто при малооблачном или ясном небе; реже из Cs или Ci\\
\textbf{Золяция} - редкое природное явление во время слабых гроз. Вид крупных или редких водяных пузырей\\

\underline{\textbf{Садки на земле/предметах:}}\\
\textbf{Роса} - капельки воды на земле/предметах из-за конден-ции пара при +t$\celsius$, слабом ветре и безоблачном небе. Может быть ночью и ранним утром: сопр-ется туманом или дымкой\\

\textbf{Иней} - белый кристал-ий осадок на змеле/предметах; аналог росы, но при -t$\celsius$. Наблюдается вечером, носью и утром; сопр-ется туманом или дымкой\\

\textbf{Кристаллическая изморозь} - белый кристал-ий осадок из мелких тонко струнных блестящих частиц льда в р-те преобразования пара на деревяьх/проводах в виде пушистых гирлянд, легко осыпа-иеся при втряхивании\\
Наблюд-ются в морозную безоблач погоду; при тумане, дымке, небольшом ветре или штиле\\
Обычно ночью, днем из-за солнца - осыпается. Обачным днем модет сохр-ся в течении суток\\

\textbf{Зернистая изморозь} - рыхлый снеговидный осадок белого цвета, обр-шийся в р-те оседания мелких капелек переохлажденного тумана на деревьях/проводах в туманную/облачную погоду при t=0..-10$\celsius$ и умер/сильн ветре\\
Укрупленные капли тумана переходят в гололед. При $\downarrow t\celsius$, слабом ветре, уменьш-ем облачности ночью, может перейти в кристалл-ую\\
Нарастание в течении тумана\\

\textbf{Гололед} (на змеле/предметах во время осадков) - плотный стекловидный лед из-за намерзания частиц осадков при t=0..-15$\celsius$. Нарастание пока есть переохлажденные осадки (неск часов). Редко при тумане и мороси - неск суток. Сохр-тся до неск дней\\

\textbf{Гололедица} (на земле после осадков)- слой обледеневшего снега или бугристого льда из-за замерзания талой воды при -t$\celsius$ после оттепели. Сохр-ется долгие дни, пока не покроится снегом или не растает из-за +t$\celsius$\\

Форма переноса осадков - метель:
\begin{itemize}
	\item \textbf{Общая:} $\geqslant 7$ м/с; сопр-ется выпадением снега\\
		При интен-ом снеге видимость $\downarrow$\\
		Возникает на атм фронтах
	\item \textbf{Низовая:} от 10-12м/с. Снег не выпадает, а переносится с поверхности снежного покрова
	\item \textbf{Поземок:} 6м/с. перенос снега над поверхностью снежного попрова
\end{itemize}
