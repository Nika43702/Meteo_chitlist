% !TEX root = ../main.tex

\section{Атм давление}
– сила, действующая на единицу горизонтальной поверхности, вызываемая весом столба воздуха, простирающегося вверх через всю атмосферу.\\
Чем больше плотность воздуха и высота столба, тем больше давление.\\
\textbf{Атмосферное давление}=высота рт. столба * удельный вес ртути. \\
Т.к. удельный вес ртути=const, то давление судят по высоте столба.
\par \textbf{Единицы измерения:}
\begin{itemize}
	\item[] \textbf{гПа} * 3/4=мм.рт.ст.
	\item[] \textbf{дюймы}=25,4 мм.рт.ст.
\end{itemize}
Нормальное давление: 1013гПа (760мм.рт.ст) на широте 45\textdegree при Т=0 [также и в СА, но Т=15]
