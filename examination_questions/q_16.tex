% !TEX root = ../main.tex

\section{Способы измерения давления}
В приборах для измерения давления используются жидкости с большим удельным весом, т.к. высота столба жидкости, уравновешивающая вес столба воздуха будет велика.\\
\textbf{Стационный чашечный ртутный барометр:} на метео станциях основной прибор для приземного слое\\
Стеклянная трубка с ртутью: верхний конец запаян, нижний погружен в чашку из 3 свинчивающихся друг с другом частей\\
Трубка помещена в металлическую оправу, на крае прорези нанесена шкала в гПа или мм.рт.ст\\
В оправу вмонтирован "термометр атташе"; по нему вводят темп поправку\\

\textbf{Металлический барометр-анероид:} деформация анероидной мембранной коробки. Внутри установлен ртутный термометр для оценки влияния T. Шкалы P и T находятся сверзу прибора\\
Измерение: Отсчитывается t (до 0,1град), отсчет давления (до 0,1 гПа), ввод поправок (шкаловая, темп-ая, добавочаня)\\

\textbf{Барограф} - для непрерывной регистрации атм P: недельный и суточный (вблизи ртутного барометра в помещении метеостанции)
