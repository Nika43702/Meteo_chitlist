% !TEX root = ../main.tex

\section{Физический смысл поправок, вводимых в отсчет изм давления}
\underline{Стационный чашечный ртутный барометр:}\\
\textbf{Инструментальная ($P_\text{н}$):} ошибки измерений из-за производства барометра. Указывается в сертификате\\

\textbf{Температурная ($P_t$)}: высота столба ртути и длина латунной шкаллы зависят от T воздуха\\
$P_t=0,000163P_\text{п}$; $P_\text{п}$ - показание барометра. При t>0 $P_t$<0; t<0 $P_t$>0\\
Рассчитывается заранее и прилагается в виде таблицы\\

\textbf{Поправка на высоту на УМ ($P_h$):} $P_h=3,14*10^{-9}P_\text{п}H$\\

\textbf{Поправка места станции ($P_\varphi$):} $P_\varphi=-0,0026Pcos(2\varphi)$; $\varphi$ - георг широта\\

$P_h+P_\varphi$ - рассчитываются заранее в одну поправку\\

\underline{Металический барометр-анероид:}\\
\textbf{Шкаловая поправка $P_\text{шк}$:} неточность изготовления прибора (передаточного механизма)\\
\textbf{Температурная ($P_t$)}: T влияет на упругие свойтсва анероида
$P_t=ct$; $c$ - темп коэф анероида. При t>0 $P_t$<0; t<0 $P_t$>0\\
\textbf{Добавочная поправка $P_\text{доб}$} - износ рычагов, поправки через интервал времени
