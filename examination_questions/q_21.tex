% !TEX root = ../main.tex

\section{Барическая ступень, использование в ГА}
\textbf{Барическая ступень} - высота, на которую нужно подняться или опуститься, чтобы давление изменилось на 1 единицу\\ 
$h=\frac{8000}{p_{cp}}(1+\frac{1}{273}t_{cp})$ [м]; при $t\uparrow$ - $h\uparrow$\\ 
В тёплом воздухе бар. ступень с уменьшением P с высотой идёт медленнее, чем в холодном \\ 
$h \sim t, \frac{1}{p}$ 

\begin{multicols*}{6} 
	H,км\\	h,м:\\ 
	0\\ 8,2\\
	5 \\ 12,9\\
	9 \\ 20,3\\
	12 \\ 29,8\\
	20 \\ 105,6
\end{multicols*} 