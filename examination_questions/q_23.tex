% !TEX root = ../main.tex

\section{Баром-ие формулы и их использование в ГА}
\begin{itemize}
	\item \textbf{Общий закон изменения давления с высотой}\\
	$P_{H}=P_{0}e^{-\frac{gH}{RT_{\text{ср}}}}$  (из уравн изотерм-ой)
	e=2,71; ($\mu=29$г/моль)\\
	$R_c=\frac{R}{\mu}$=286,5 Дж/(кг*К); далее везде R\\
	\item \textbf{Барометрическая формула Лапласа:}  используется до 11км;\\
	не учитывает влажность, широту места и изменения g, зависящая от H\\
	$ H=18400(1+\frac{1}{273} t_\text {ср})\log \frac{P_{0}}{P_{H}}$
	\item \textbf{Формула Бабине:} для небольших разниц высот (менее 1км)\\
	$h$=$H_{H}-H_{0}$=$16000(1+\frac{1}{273} t_\text {ср})\frac{P_{0}-P_{H}}{P_{0}+P_{H}}$ (из уравн однородной)
\end{itemize}
\underline{\textbf{Использование:}}
\begin{itemize}
	\item Барометрическое нивелирование (разность высот поверхности) – превышение одного пункта над другим по наблюдениям давления на этих уровнях и средней температуры рассматриваемого слоя.
	\item Рассчитать барометрическое давление на заданной высоте, если известно давление на уровне моря и средняя температура слоя; или высоту, зная давление и среднюю температуру.
	\item Приведение давление к уровню моря, т.е. найти P0 ниже лежащего уровня, совпадающего с уровнем моря, зная давление на высоте и высоту над уровнем моря.
\end{itemize}
\par Приведение давления к уровню моря производится на всех станциях, оно наносится на карты погоды, позволяя сравнивать между собой величины давления во всех пунктах земного шара.