% !TEX root = ../main.tex

\section{Основные формы бар поля, опр-ие их по картам погоды}
\textbf{Изобарические поверхности} - поверхности с одинаковым давлением. Располагаются одна над другой, не параллельны УМ\\ 
Повышаются над областями тепла и понижаются над областями холода\\ 
\textit{Высота изобарической поверхности} измеряется от УМ, измеряется в геопотенциалах\\

\par \textbf{Геопотенциальный метр} численно равен работе, затраченной на подъем 1т воздуха на 1м при g=9,8 $m/c^2$. Характеризует качественную сторону\\ 
\textbf{Линейный метр} характеризует количественную сторону. 
Гпм отличается от лин. метра на 0,3\%\\ 
$H_{\text{гпм}}=67,44T_{cp}lg\frac{P_{0}}{P_{H}}$\\

\par \textbf{Изобара} - пересечение изобарических поверхностей с УМ\\
\textbf{Барическое поле:} картина давления, где бар поверхности наносятся на карту погоды каждые 5гПа, показывающая распределение давления воздуха\\
\textbf{Формы бар. поля:} 
\begin{itemize} 
	\item \textbf{Циклон (барический мин)} - область пониженного P, ограниченная замкнутыми изобарами с наименьшим P в центре.\\ 
	\textit{Ветер:} против ч.с. к центру
	\item \textbf{Антициклон (бар макс)} - область повышенного P, ограниченная замкнутыми изобарами с наибольшим P в центре.\\ 
	\textit{Ветер:} по ч.с. из центра 
	\item \textbf{Ложбина} - вытянутая область пониж. P, вдоль которой можно провести ось ложбины. (Не замкнутые изобар. поверхн) 
	\item \textbf{Гребень} - вытянутая область повышен. P, вдоль которой (по наибольшим p) можно провести ось. (Не замкнутые изобар. поверхн) 
	\item \textbf{Седловина} - бар. поле между двумя крест накрест лежащими двумя областями циклона и антициклона 
\end{itemize} 

\textbf{Условия погоды:} 
\begin{itemize} 
	\item[] Плохие: циклон, ложбина 
	\item[] Благоприятные: антициклон, гребень 
\end{itemize} 

