% !TEX root = ../main.tex

\section{Давление QFE,QNH,QFF и их использование при обеспечении полетов}
\par \textbf{QFE} – давление на уровне а/д или порога ВПП
(дается экипажу для обеспечения посадки)\\
\textbf{QFF} – QFE, приведенное к УМ с учетом реальной температуры\\
\textbf{QNH} – QFE, приведенное к УМ по условиям СА\\
\textbf{QNE} – давление СА (760 мм.рт.ст/1013 гПа)

\textbf{Эшелон} - выделяемое для полётов относительная барометрическая высота, отсчитываемая от изобарической поверхности 1013 гПа.(фактически полет по барометрическим поверхностям)

Чтобы найти QNH:\\
1. По таблице СА определяем стандартную баром-ую высоту $H_p$ (Зная $P_h$)\\
2. $\varDelta=H_p-H$; H-на аэродроме\\
3. По таблиице СА опр-яем QNH по $\varDelta H$ [округляем до меньшего значения гПа (для данных в сводках)]