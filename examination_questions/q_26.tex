% !TEX root = ../main.tex

\section{Пространственно-временная изменчивость давления, используемая в практике характеристики}
\par \textbf{Скорость изменения давления} 
\begin{itemize}
	\item земли: несколько гПа/сутки
	\item При приближении фронтов и циклонов: 1-2гПа/ч и до 25гПа/сутки
	\item С приближением тайфуна был макс: 24гПа/час
\end{itemize}
С высотой давление уменьшается быстрее в нижних слоях, чем на высоте\\
\par \textbf{Изменение давления с высотой:}\\
5 км  ½ давления у земли\\
10км ¼ давления у земли\\
20км $\frac{1}{20}$ давления у земли

\textbf{Суточный ход:} в основном непериодичный\\  
Суточная амплитуда в среднем 3-4гПа\\ 
В тропиках: 2 максимума и 2 минимума\\ 
К 60 широтам: уст амплитуда-десятые доли гПа\\ 
\textbf{Мин:} утром/после обеда \\
\textbf{Макс:} перед полуднем/полуночью \\
\textbf{Годовой ход:} 
\begin{itemize} 
	\item \textit{Умеренные широты:} \\ 
	Над океанами: циклоны глубже зимой, чем летом\\ 
	Над материками: антициклоны-зимой, циклоны-летом 
	\item \textit{Субтропики:} \\ 
	Над океанами: антициклоны в Северном полушарии: в июле; в Южном: в январе
	\item \textit{Экватор:} циклоны весь год 
	\item \textit{Над Арктикой:} антициклоны, сам антициклон над Гренландией 
	\item \textit{Над Антарктикой:} антициклон устойчивый весь год 
\end{itemize} 

\par \textbf{Типы годового хода:} 
\begin{itemize} 
	\item Над материками: max-зимой; min-летом. Амплитуда увел с удавлением от океана 
	\item \textit{Высокие широты:} max-летом; min-зимой 
	\item \textit{Средние широты:} max-летом и зимой; min-весной и осенью 
	\item \textit{Тропики:} годовой ход выражен слабо 
\end{itemize}