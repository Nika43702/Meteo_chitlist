% !TEX root = ../main.tex

\section{АД, устройство, построения}
\textbf{Кривая состояния} - графическое изображение адаб измен-ия T в подним-ейся воз массе при любых T и P: нижняя часть до уровня конденсации - сухая адиабата; выше - влажная адиабата\\
\textbf{Адиабата} - графичекий способ изображения изменения T при адиаб процессах\\
\textbf{Кривая стратификации} - кривая, показывающая распределение температуры с высотой над а/д\\
\textbf{Изограмма} - линии одинаковой удельной влажности (под откос вправо)\\
\textbf{Депеграмма} - кривая точек росы по высотам\\

Если ширина между кривыми состояния и стратификации неустойчивой атм-ы:\\
 4-6 градуса - умеренная \textbf{турбулентность}\\
 >6 градуса - сильная\\
 
\textbf{Конденсационный слой:} в самом верху ищем изограмму со значением 0,15 (между 0,1 и 0,2), проводим линию вниз/влево до пересечения с кривой стратификации - между изограммами 0,15-0,1 - конденсационный слой\\

\textbf{Слой обледенения:} строим кривую показывающую темп насыщения надо льдом: $T_\text{нл}=-8(t-td)$, та область где кривая стратификации левее и T=[(-40)-0]- слой обледенения
 
 