% !TEX root = ../main.tex

\section{Состав атмосферы}
\textbf{Атмосфера} – воздушная оболочка Земли. Представляет собой газовую смесь, от 0 до 100км состав однородный. Нижняя граница атмосферы - поверхность Земли; выраженной верхней границы нет, но на 1300км плотность газов уменьшается на столько, что приближается к значению межпланетного пространства. \\
\textbf{Состав:}
{
	\setlength{\columnseprule}{0pt}
	\begin{multicols*}{2}
		\begin{itemize}
			\item[] N-78\%
			\item[] O$_2$-21\%
			\item[] Аргон-0,9\%
			\item[] CO$_2$ 0,03 \%
			\item[] +(Н$_2$0) и Озон(O$_3$)
			\item[]	Неон
			\item[] Гелий
			\item[] Криптон
			\item[] Ксенон
			\item[] Водород - <0,1\%
		\end{itemize} 
	\end{multicols*}}
\textbf{Масса атмосферы:}\\
До 5км 50\% \hspace{7mm} До 20км	90\% \\
До 10км	75\% \hspace{6mm} До 30-35км	$\approx100$\% \\
\textbf{\textit{Углекислый газ(CO2)}} поглощает инфракрасную радиацию, уходящую от поверхности З.\\
\textbf{\textit{Озон (O3)}} улавливает значительную часть ультрафиолетного излучения (совершенно не пропускает короткие лучи) и прогревает отдельные слои от 0 до 50км.\\
Максимальная концентрация на высоте 20-25км\\
Определяет температурный режим стратосферы, т.к. поглощает излучение уже на 50-55км, вследствие чего температура поднимается.

\textit{\textbf{Водяной пар:}} попадает в атмосферу с поверхности суши, рек, морей и океанов. Поглощает тепло от солнца днем/от Земли ночью, предохраняет поверхность З от перегрева днем, от переохлаждения ночью.\\
\textit{\textbf{Примеси}} (пыль, кристаллы соли, промышленный дым): играют важную роль в изменении режима формирования облачности