% !TEX root = ../main.tex

\section{Сухадиаб и влажад градиенты}
\textbf{Сухоадиабатический градиент} (мера охлажд/ нагрев-ия сухого воздуха при ад процессе) - изменение темп в данном объеме сухого/ ненас-ого водянными парами воздуха при поднятии/ опускании его на каждые 100м.\\ $\gamma_a=0,98\celsius/100\text{м}(\approx1\celsius/100\text{м})$\\
\textbf{Влажноадиабатический градиент} - для воздуха насыщенного водяным паром. При поднятии воздуха часть пара конденсируется, выделяется скрытое тепло конденсации, уменьшающее величину охлаждения.\\
Влажноад градиент, зависящий от T и P, обычно меньше сухоад-ого.\\
Чем выше T в подним насыщ воздухе, тем меньше величина влажноад градиента. Т.к. при более высокой T в насыщ-ом воздухе содер-ться большое кол-во водяного пара, при конденсации которого выделяется большое кол-во скрытого тепла, в рез-те воздух охлажд медленее.\\
С умень-ем P (T=const) влажноад градиет уменьшается, т.к. воздух менее плотный, и скрытое тепло идет нагревание меньшей массы воз-ха\\
При опускании насыщ воздуха происходит адиаб нагревание, в рез-те воз-ух удаляется от состояния насыщения. Т.е. опускающ-ся воздух всегда нагревается по сухоад закону