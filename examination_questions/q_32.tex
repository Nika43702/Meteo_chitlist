% !TEX root = ../main.tex

\section{Верт движения, условия формирования}
\begin{enumerate}
	\item \textbf{Термическая конвекция} - вид верт. движ, возникающий вследствие неравн-ого нагревания воз-а от подстилающей поверх-и.\\
	Над более нагретыми участками поверх-ти воздух быстро прогревается, становится теплее окр.воздуха (т.к. более легкий) поднимается	вверх. Рядом с таким восходящим потоком появляется нисходящий. \hspace{7ex} \underline{Виды:}\\
	*\textit{Неупорядоченные токи воздуха} (термическая турбулентность)\\
	*\textit{Мощные упорядоченные движения больших масс воздуха}\\
	Скорость может достигать нескольких м/с, иногда в Cb и более 20-30 м/с
	\item \textbf{Динамическая конвекция} (динамическая турбуленность) - неупорд вихревые движ-ия в слое до 1,5 км, возник-ющие при гор перемещении и трении воздуха о подстилающую поверхность.\\
	Вертик состав-ая - от неск десятков см/с до м/с
	\item \textbf{Вынужденные вертикальные движения}\\
	*\textit{Упорядоченные восходящие скольжения:} при натекании теплого воздуха по клину холодного (теплый фронт). Скорость: несколько см/с, гор протяж-ть до несколько тысяч км\\
	*\textit{Вертикальные движения воздуха:} при активном подклинивании холодного воздуха под теплый (холодный фронт) и при встречи воз потока с крупными препятствиями
	\item \textbf{Волновые движения воздуха:} на слоях инверсии (ниж/верх границы) вследствие разности плотности и скорости движ воздуха над/под инверсией. Также наблюдаются над горами на их подветренной стороне (подветренные и стоячие волны). Скорость не превосходит нескольких м/с
\end{enumerate}