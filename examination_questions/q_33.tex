% !TEX root = ../main.tex

\section{Условия устойчивости и неустойчивости}
Начальное возн-ие верт-ых движений: см 32\\
Дальнейшее развитие зависят от соотношения T подним/опуск объема воздуха и Tокр воздуха\\
\textbf{Случаи развития верт движений:}
\begin{enumerate}
	\item \textbf{Неустойчивое равновесие воздуха:}\\
	Если верт градиент $T_\text{окр.возд}$ > сухоад-ого и влажноад-ого, то массы возд-ха становятся неустойчивыми (т.е. будут подниматься/опускаться)
	\item \textbf{Устойчивое равновесие}\\
	Если верт градиент $T_\text{окр.возд}$ < влажноад-ого и сухоад-ого, то массы будут возращаться в прежнее устойчивое состояние
	\item \textbf{Влажно-неустойчивое равновесие} (влажнонеустойчивость)\\
	Если верт градиент $T_\text{окр.возд}$ < сухоад-ого, но > влажноад-ого, то только при подъеме насыщ воздуха будет неустойчивое равновесие
	\item \textbf{Безразличное равновесие}\\
	Если верт градиент $T_\text{окр.возд}$= сухоад-ому или влажноад-ому, то некоторый объем воздуха, поднятый/опущенный на какую-л. высоту, здесь и останется
\end{enumerate}
В насыщ воздухе восходящие движения возникают легче, чем в сухом. При одном и том же верт градиенте насыщенный вохдух всегда более неустойчив\\
При неустойчивом состоянии верт движения воздуха интенсивно развиваются. Начавшееся движ-е воздуха вверх/вниз продолжается с возрастающей скоростью.  Полет сопровождается бросками самолета и болтанкой\\
Слои, препятсвующие развитию верт движений - \textbf{инверсии}, т.к. с высотой темп увеличивается, то подним воз масса, дойдя до этого слоя прекращает движение
