% !TEX root = ../main.tex

\section{Классификация туманов}
\textbf{По видимости:}
\begin{itemize}
	\item Сильные: <50м
	\item Умеренные: 50-500м
	\item Слабые: 500-1000м
	\item Дымка: >1000м
\end{itemize}
\textbf{Мгла} - взвесь твердых частиц, ухудшающая видимость\\
\underline{\textbf{Внутримассовые:}}\\
\textbf{- Туман охлаждения:}
\begin{itemize}
	\item \textbf{Радиационные:} вследствие радиац-ого выхолаживания зем пов-сти и охлаждения от неё призменого слоя\\
	
	\textit{В темплу половину года:} ночью в ясную/малооблачную погоду при ветре $\leqslant 3$м/с; над низинами и заболоченными местами\\
	С восходом солнца рассеив-ся\\
	Верт. мощность: от неск м до неск 10м\\
	Гор. видимость: до 100м и менее\\
	
	\textit{В холодную половину года:} более опасный, т.к. при ясной погоде выхолаживание вследствие непрер-ого излучения в течение ряда дней распр-ется на большую высоту\\
	Верт. мощность: до 1,5-2км и продолж-ое t
	\item \textbf{Адвективные:} при движ-ии теплых влажных масс по холодной продсти-щей повер-и\\
	Приземный слой охлажд-ся и путем турб-ого перемеш-ния охлаж-ие распр-ется до высоты в неск 100м\\
	Сопровождается нередко моросящими осадками\\
	При U$\geqslant5-10$м/с в любое t суток и могут длительно сохр-ся (до неск дней)\\
	
	\textit{Над материком:} в хол-ую половину года при движ теплых влж морских масс по выхложенной повер-и почвы/движ масс с более теплых участок на более холодные\\
	в тепл половину года: при движ теплого с суши на холодное море\\
	
	\textit{Над морем:} в течение всего года при движ-ии с более теплой морской поверх-и на более холодную
	\item \textbf{Адвективно-радиационные:} при совместном действии адвекции и радиац-ого охлаждения
	\item \textbf{Туманы склонов:} в горах, когда при подъеме по склонам охлаждается вследствие адиаб расщирения (темп$\downarrow$ до точки росы)
\end{itemize}
\textbf{- Туман испарения:}  в рез-е притока пара за счет испарения с теплой водной повер-сти в более холодный воздух (разн темп воды и воздуха>10)
\begin{itemize}
	\item \textbf{Морские:} над незамерз-щими заливами, полынями в зимние месяцы
	\item \textbf{Осенние:} над реками и озерами осенью, когда поверх-ть воды теплее, чем воздуха
\end{itemize}
\underline{\textbf{Фронтальные:}} повяляется в рез-те конденсации пара, близком к насыщ-ию из-за испарения выпадающих осадков при фротнах. P$\downarrow$ перед теплым фронтом приводит к адиаб расш-ию приземного воздуха и его охлаждению\\
Ширина до 200км; опасен при сливании с облаками
