% !TEX root = ../main.tex

\section{Методы, средства изме видимости}
На полосе может опр-тся с помощью огней на ВПП: боковые посадочные огни (каждые 60м) (свыше 600м) и осевые огни (15-30м) (до 350м), по обоим огням  от 350-600м\\

\textbf{Визуальные наблюдения:}
Дневные: выбирают 9 темных объектов, на расстоянии 50,200,500м и 1,2,4,10,20 и 50 км от пункта наблюдения
Ночные: по одиночным источникам света опр-ой силы, расположенным на различных расстояниях (9шт). Видимые огни - светязиеся точки, а не расплывчатые пятна

\textbf{Инструментальные:} Регистратор видимости РДВ-2 для дискретных и непрерывных наблюдений дистанционных измерений прозрачности атмосферы в любых метео условиях, днем и ночью

\underline{Приборы:} \textbf{Фотометрический блок:} сравнение двух световых потоков полученных от одного и то же источника\\
\textbf{Импульсный фотометр:} для дискретных и непрерывных дистанционных наблюдений прозрачности атм-ы