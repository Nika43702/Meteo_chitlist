% !TEX root = ../main.tex

\section{Основные принципы и подходы к разделению атм на слои}
\textbf{1. По составу:}
\begin{itemize}
	\item \textit{Гомосфера:} Постоянный состав чистого сухого воздуха до 100-120км вследствие интенсивного перемешивания
	\item \textit{Гетеросфера:} Выше 120км атмосфера в основном состоит из 2 газов: Азота и Кислорода, чьи молекулы диссоциированы (разложены на атомы)
	\item \textit{+Выше 1000км} только Гелий, еще выше-только Водород
\end{itemize}

\textbf{2. По взаимодействию с поверхностью З:}
\begin{itemize}
	\item Приземный 	0–50-100м  (резкое изм P,T,f,U)
	\item Пограничный	0 – 1-1,5км
	\item Свободная атм: пренебрегаем силу трения
	\begin{itemize}
		\item Средний:1,5-6км, погодные условия определеяются атм фронтами и верт токами
		\item  Высокий: до тропопаузы, слоистые облака, струйные потоки
	\end{itemize}
\end{itemize}

\textbf{3. По взаимодействию с летательными аппаратами:}
\begin{itemize}
	\item Плотный 	до 150км
	\item Околоземное космическое пространство	выше 150км
\end{itemize}

\textbf{4. По температуре:}
(5 вопрос)