% !TEX root = ../main.tex

\section{Полетная видимость}
- видимость определяемая с самолета. Может быть:
\begin{itemize}
	\item \textbf{Горизонтальная:} видимость объектов на уровне полета
	\item \textbf{Вертикальная:} видимость объекта на земной поверхностю
	\item \textbf{Наклонная:} (наиболее важная); - максимальное расстояние, на котором с высоты можно увидеть предмет на земле под разными углами\\ 
	Частный случай: \textit{посадочная} - предельное расстояние по наклону вдоль глисады снижения, на котором летчик, совершающий посадку, может обнаружить и распознать ВПП\\
	
	Посадочная и наклонная могут отличаться, т.к. под самолетом могут быть задерживающие слои (изотермия, инверсия), под которыми скапливаются водяной пар и др частицы. 
	В результате прям над полосой (или по наклонной ближе в вертикальной) полосу видно хорошо (самое малое расстояние через замутненный слой), а по глиссаде, пилот смотрит через большее расстояние замутненного слоя и полосу может не видеть\\
	В среднем наклонная при НГО до 100м - 30\%; 100-200м 50\% от гориз видимости\\
	При высоте облаков выше 200м, примерно равна горизонтальной\\
	Может быть измерена через измерение прозрачности под наклоном по глиссаде (инструментально)
\end{itemize}