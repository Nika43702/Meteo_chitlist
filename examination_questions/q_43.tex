% !TEX root = ../main.tex

\section{Облака НЯ и влияние их на полеты ВС}
\textbf{Облака} – существенный погодообразующий фактор, определяющий формирование и режим осадков, влияющих на тепловой режим атмосферы и Земли.\\
Перемещаются на тысячи км, перенося и перераспределяя огромные массы воды.\\
см 46 вопрос о нижнем ярусе\\

\par \textbf{Явления, опасные для полета:} 
\begin{itemize}
	\item Турбулентность: вызывает болтанку
	\item Вертикальные токи: сильные броски ВС
	\item Грозовые явления, шквалы, обледенение, ливневые осадки, град и т.д: осложняют полет/посадку и взлет
	\item Низкая облачность: осложняет посадку
\end{itemize}

\par \textbf{Трудности полета в облаках:}
\begin{itemize}
	\item Отсутствие визуальной ориентировки
	\item Ухудшение видимости
	\item Пилотирование только по приборам
	\item Обледенение при T<0, влияющее на аэродинамические свойства самолета.
	\item Болтанка из-за турбулентности
\end{itemize}