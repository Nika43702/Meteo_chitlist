% !TEX root = ../main.tex

\section{Морфологическая классификация облачности}
\textbf{Морфологическая (форма):}
\begin{itemize}
	\item \textit{\textbf{Верхний Ярус:}} 6км – до тропопаузы; кристаллы\\
	Ci Перистые \\
	Cc Перисто-кучевые \\
	Cs Перисто-слоистые \\
	Опасны метловидные вершины над Cb; облачные полосы, связанные с сильными ветровыми потоками струйных течений\\
	Движение  с Запада Ci и Cs – признак ухудшения погоды и выпадения осадков.\\
	Наличие плотных Ci при высокой T и d – неустойчивость атмосферы и возможность гроз (теплый фронт)
	
	\item \textbf{\textit{Средний Ярус:}} 2 -6 км; Кристаллы и смешанные\\
	Ac Высококучевые\\
	As Высокослоистые\\
	Башенкообразные и хлопьевидные As – неустойчивость состояний атм-ы и болтанка; предвещают развитие гроз;
	Высококучевые чечевицеобразные (Lenticularis) предвещают cold фронт; в горах – предветренные стоячие волны
	Уплотнение As – приближение  warm фронта с обложными осадками; возможно обледенение в As
	
	\item \textbf{\textit{Нижний Ярус:}} Ниже 2 км; Капли Зимой: переохлажденные капли и кристаллы со снежинками\\
	Sc Слоисто-кучевые\\
	St  Слоистые\\
	Ns Слоисто-дождевые\\
	Cu Кучевые (кроме мощных)\\
	Ns вертикальная мощность (неск км) размытая н.г. из-за осадков\\
	Зимой: ВНГО ниже 100м. Под Ns – дождевые облака (Cu Fr) {\hspace{1cm} В Ns T<0 обледенение}\\
	Летом: Ns выше, дождевые облака реже.\\
	В Ns грозовые очаги, чаще ночью\\
	St и Sc изменчивость ВНГО (<300м)\\
	Sc часто внутримассовые особенно зимой, испытывают волновые движения
	
	\item \textbf{\textit{Вертикального развития:}} НГО-ниж. яр; ВГО-ср/верх яр\\
	Cu Congestus Кучевые мощные\\
	Cb Кучево-дождевые\\
	В области Cb: сильная болтанка, верт броски на сотни метров, сильное обледенение. Особенно опасна передняя часть Cb, под основанием «крутящий вал» с горизонтальной осью (шквальный ворот)
\end{itemize}