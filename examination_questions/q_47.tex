% !TEX root = ../main.tex

\section{Облака, условия формирования, микроструктура, влияние на полеты ВС}
\textbf{Причина облаков:} адиабатическое понижение t в поднимающемся влажном воздухе, приводящая к конденсации водяного пара. Кроме этого, понижение t происходит в следствии радиационного выхолаживания верхних слоев инверсий или верхней границы облаков.

\par \textbf{Причины подъема воздуха:}
\begin{itemize}
	\item Конвекция – вертикальное движение воздуха
	\item Вынужденное вертикальное движение
	\item Волнообразное движение
	\item Турбулентность
\end{itemize}
\textbf{Условие для образования облаков:}
\begin{itemize}
	\item Достижение состояния насыщения водяным паром по отношению к той поверхности, на которой начинаются конденсация или сублимация
	\item Наличие в атмосфере ядер конденсации, на которых конденсация водяного пара может начаться при относительной влажности менее 100%
\end{itemize}

\textbf{По микрофизическому строению:}\\
В результате водяного пара образуются:

\begin{itemize}
	\item \textit{Мелкие капли} (диаметр около 50мкм (микрон))
	\item \textit{Ледяные кристаллы} (вид 6-и гранной призмы)
\end{itemize}

Водность - кол-во воды в г в 1куб метре\\
\par \textbf{Облака:}
\begin{itemize}
	\item \textbf{Капельножидкие} (из капель): водность: 0,01-4 (гр/м3)\\
	T>-12 в общем случае, но до -40 капли находятся в переохлажденном состоянии
	\item \textbf{Кристаллические/ледяные:} T<-40; водность: <0,02 (гр/м3)\\
	T<-40
	\item \textbf{Смешанные:} -12<T<-40; водность: $\leqslant$0,2-0,3 (гр/м3)\\
\end{itemize}

+см 43 вопрос:
