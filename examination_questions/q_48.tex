% !TEX root = ../main.tex

\section{Силы, действующие в атм, опр-щие скорость и напр ветра}
В слое трения (0-1000м) при установившемся движении сила Барического градиента уравновешивается суммой сил Кориолиса и трения. $\vec{F}_{p}=\vec{F}_{k}+\vec{F}_{\text{тр}}$\\
\textbf{Барический градиент} - изменение P вдоль земной поверхности: $\text{Г}_{p}=\frac{\varDelta P}{\varDelta S}$\\
За ед S берется 1$\textdegree$ дуги меридиана (111км=60nm)\\
Средняя величина бар градиента: 1-2гПа на 111км, наибольшее: 15-20гПа (ураган)\\
\par \textbf{Сила бар градиента}(бар градиент к массе) - сила, заставляющая массу воздуха прийти в гор движение. $F_{\text{гр}}=\frac{1}{\rho}\frac{\varDelta P}{\varDelta S}$ (см/$c^2$)\\
Воздух перемещается по напр-ию вектора $F_{\text{гр}}$; возникает $F_{\text{тр}}$ (до 1 км на землей) и $F_k$, а также центробежная чила (при криволинейном двжении; НО траектория движ-я имеет небольшую кривизну (до 1000км у циклона и антициклона), поэтому центр-ой силой можно принебречь)
\par \textbf{Сила Кориолиса - $F_k$} - инерциальная сила, которая возникает вследствие суточного вращения Земли вокруг своей оси.\\
Воздух отклон-ся, т.к. он по инерции сохр-ет свое напр-ие движ-я отн-но мирового простр-ва в то время, как Земля поворачивается.\\
$F_k$ действует под прямым углом к напр движ воздуха и влияет на напр-ие (вправо в сев полушарии); не влияет на скорость!\\
Значение зависит от U и $\varphi$ местности: $F_{k}=2\omega usin\varphi$
\par \textbf{Сила трения} возникает в рез-те терния воздуха о неровности поверхности и напр-а против-но движ-ю воздуха: $F_\text{тр}=\mu U$; $\mu$ - коэф трения\\
Изменяет U и напр ветра. С высотой убывает, выше 1км практически не влияет.
