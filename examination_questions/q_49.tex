
% !TEX root = ../main.tex

\section{Изменение ветра с H в слое трения}
\par \textbf{Барический закон ветра или правило Бейс-Балло:} В слое трения если встать спиной к ветру, то P-1 находится слева и чуть впереди, а P+1 справа и позади, т.к. воздух отклоняется вправо и пересекает изобары
\par\textbf{Угол отклонения ($\alpha$)}- угол между вектором $F_{\text{гр}}$ и U. Зависит от коэф. трения (k) и $\varphi$ места.\\
Над сушей из-за большей $F_{\text{тр}}$ угол отклонения  $\approx30-45\degree$, над морем ($F_{\text{тр}}$ меньше) - может быть $\approx90\degree$. \\
На экваторе движение совпадает с напр $F_{\text{гр}}$; самое большое отклонение на полюсе на широте 90\textdegree\\

\par В приземном слое $F_{\text{тр}}$$\downarrow$ с H, U$\uparrow$ и поворачивает в $\longrightarrow$ пока не станет \textbf{градиентным} (спираль Экмана). На высоте 500м U в 2 раза больше.