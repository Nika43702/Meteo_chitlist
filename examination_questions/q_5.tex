% !TEX root = main.tex

\section{Деление атм-ы на слои по распр темп}
\begin{itemize}
	\item\ \textbf{Тропосфера} 	80\% массы атмосферы, 90\% водяного пара. Западный перенос\\
	Толщина: в среднем 12км (в СА 11км)\\
	Источник тепла – земная поверхность, нагреваемая солнцем.\\
	T↓с высотой, градиент=0,65\celsius/100м [2\celsius /1000ft]. \underline{\textit{Слои:}} 	
	\begin{itemize}
		\item\ \textit{Пограничный} 		до 1,5км
		\item\ \textit{Средняя тропосфера} 	1,5-6км
		\item\ \textit{Верхний слой}		6км-Тропопауза (верхняя граница)
	\end{itemize}
	\textit{Тропопауза}: задерживающий слой; T=const
	Высота зависит от места и T. (Чем T↓, тем H↓)\\
	На полюсах в ср: 8км; T=-40\celsius (Л) или -60\celsius (З)\\
	На экваторе с ср: 16км; T=-80\celsius\\
	Толщина: некс 100м-(2-3)км
	\item \textbf{Стратосфера} 	До 50км вместе с тропосферой составляет 99\% всей массы атмосферы\\
	T сначала не изменяется, потом ↑ до 50км; ср значение 0 (может достигать 20\celsius)\\
	Рост обусловлен взаимодействием ультрафиолетового излучения с озоновым слоем.\\
	Содержание вод пара↓, содержание озона↑\\
	Могут наблюдаться перламутровые облака на высоте 20-30км (Хотя воздух сухой)\\
	Верхняя граница – \textit{Стратопауза.} T=const до 51км.\\
	\item \textbf{Мезосфера}
	Высота от 51км-80км\\
	T↓, в ср до -80\celsius\\
	\textit{Можно наблюдать:} 	
	\begin{itemize}
		\item вспышки метеоров
		\item Серебристые облака (на высоких слоях)
	\end{itemize}
	Заканчивается \textit{Мезопаузой} (T может быть от -120 \celsius до -50 \celsius) T=const. 
	\item \textbf{Термосфера} H=80-800км; T↑ до 1000К (727 \celsius)\\
	\textit{Ионосфера} до 1000км. Северное сеяние\\
	\textit{Экзосфера} выше 1000м\\
	Т.к. воздух разряжен выше 1000км, то температура принимается не как степень нагретости тела, а как мера средней скорости разгона частиц до столкновения друг с другом. Частицы (V=11,2км/с) вылетают в космическое пространство; на высоте 800км (T=1000K) атмосфера переходит в межпланетное пространство.
\end{itemize}
