% !TEX root = ../main.tex

\section{Изменение ветра с высотой в свободной атм}
Выше слоя трения $F_{\text{гр}}$ и $F_k$, направленны в $\leftrightarrows$ стороны и $\perp$ к изобарам, они уравн-ют друг друга: воздух движется вдоль изобар. [низкое P - слева]\\
Скорость опр-ется $F_{\text{гр}}$, [$F_k$ не влияет], поэтому поток - \textbf{Градиентный ветер}.\\
Скорость может как убывать, так и возрастать, могут быть как правые, так и левые повороты ветра. Обычно U возрастает и максимальна у тропопаузы, выше убывает. Под тропопаузой могут быть  очень сильные потоки U>30м/с (струйные течения)
$F_p=F_k$ \hspace{5ex} $ \frac{1}{\rho}\frac{\vartriangle P}{\vartriangle S}=2\omega u_{\text{гр}}*sin\varphi $\\
$ u_{\text{гр}}=\frac{1}{\rho 2\omega sin\varphi}\frac{\vartriangle P}{\vartriangle S} $, $\rho=\frac{P}{RT}$\\
$ u_{\text{гр}}=\frac{RT}{P 2\omega sin\varphi}\frac{\vartriangle P}{\vartriangle S} $,
пусть $K=\frac{RT}{\rho 2\omega}$, тогда формула для опр-ия скорости градиентного ветра для H=500-1000м (где сила трения перестает влиять) по приземным картам погоды:\\
$ \upsilon_{\text{гр}}=\frac{K}{sin\varphi}\frac{\vartriangle P}{\vartriangle S} $;
при $t=0\celsius, P=1000$гПа: $ \upsilon_{\text{гр}}=\frac{536}{sin\varphi}\frac{\vartriangle P}{\vartriangle S} $ м/с