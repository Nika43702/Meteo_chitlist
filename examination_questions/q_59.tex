% !TEX root = ../main.tex

\section{Реальный ветер и влияние на полеты}
Ветер не постоянен и нестойчив, порывистый из-за турбулентности.\\
Хар-ки приземного ветра влияют на взлет/посадку\\
ветер на высотах - на нав. элементы полета.\\
При сильном ветре на а/д могут возникать: метели и пыльные бури, которые ухудшают гор. видимость ниже минимума погоды а/д.\\
Ураганы и шквалы при взлете/посадке могут приводить к летным происшествием.\\
Турбулентность атмосферы вызывает интенсивную болтанку и броски ВС.\\
Для обеспечения безопасности полетов и выполнения их по расписанию ветер учитывается при всех нав расчетах.\\
Климатические хар-ки ветра учитываются при строительстве а/м и составлении расписания движения на воз трассах.\\ 
При выполнении гор полета ветер влиянет на W и УС при V=, от напр-я и U ветра зависит продолжительность полета по воз трассе, при боковом ветре ПУ будет отличаться от КУ.\\
Ветер влияет на взлетно посадочные хар-ки: длину разбега, V отрыва, длину пробега и посадочную V. Наиболее благоприятным для взлета и посадки является встречный ветер, т.к. все перечисленные взлетно-посадочные хар-ки имеют меньшую величину и самолет имеет лучшую устойчивость и управляемость.