% !TEX root = ../main.tex

\section{Методы, средства измер-я W у Земли}
Скорость и напр-ие изме-ются на высоте 6-12м от поверхности ВПП\\
Через 2 мин обновление для взлета, посадки и руления\\
10 мин для подхода и круга\\
Порывы за последние 10минут\\
Сдвиг ветра каждые 30метров\\

\underline{\textbf{Первичные приборы:}}
\textbf{Флюгарка:} жесткая асимметричная система из пластин и противовеса, свободно вращающаяся отн-о верт оси\\
\textbf{Чашечная вертушка:} 3/4 полые полусферы на центральной втулке, радиаьно расположены в одно плоскости перпендикулярной оси\\
\textbf{Воздушный винт:} >=3 лопастей различной формы и размеров\\

\underline{\textbf{Используемые приборы:}}
\textbf{Анемометр Фусса (U):} Приемник 4-хчашечная вертушка, ее ось соединена с мех счетчиком; устанавляивается на 2м от земли. Включают - записывают показания; через 20-30сек влючают секундомер и через 2/10мин снимают показания: получают среднее число оборотов в секунду; по таблице оценивают ср скорость ветра\\
Погрешность: $\pm(0,3\pm0,05U)$м/с\\
\textbf{Анемометр ручной индукционный (АРН-49) (U)} 3-х чашечная вертушка с магнитным тахометром. Магнит помещен в металлический стакан;  к нему крепится стрелка прибора\\
Погрешность: $\pm0,5\pm0,05U$ м/с\\

\textbf{Анемометр WAA151 (U):} для дистанционного измерения U у земли. Погрешность: $\pm$м/с\\
\textbf{Флюгер WAF151:} оптикоэлектронный флюгер с малым порогом чувствительности. Погрешность: $\pm3\degree$\\

\textbf{Дистанционная метео станция (ДМС)} для d,U,t и f(отн влажность). Блок дотчикв для ветра: 8лопастной воз винт, преобразователь - тахогенератор переменного тока. Чувствительный элемент - флюгарка, преобразователь - бесконтактный сельсин (эл. машина переменного тока)\\
Погрешность: U=$(\pm0,5+0,05U)м/с$; d=$\pm10\degree$; t=$\pm1$; f=$\pm7$\%\\

\textbf{Анеморумбометры}: для дист-ого измерения: d, ср, min и max U. Погрешность: разная для каждого параметра и для каждой модицикации разный.




