% !TEX root = ../main.tex

\section{Измерение темп у Земли и по высотам}
На метео площадке в 2 психометрических будках (2м над почвой)\\
В 1: Стационный психометр\\
Максимальный/минимальный термометры\\
Гигрометр\\
Во 2: Термограф и Гигрограф\\
Если будок нет, то используют аспирационные психометры в полевых условиях\\
\underline{\textbf{Термометры:}}\\
\textbf{По принципу действия:}\\
\textit{Жидкостные} (спиртовые, ртутные): точность до 0,2-0,5\\
\textit{Металлические:} сопротивления, биметаллические пластины/спирали: изм размера тел\\
\textit{Полупроводниковые:} изм эл. сопротивления металла\\

\textbf{По сроку действия:}\\
\textit{Срочный:} (вертикальные) - t в момент наблюдения. Ртутные или спиртовые\\
\textit{Максимальный:} Ртутный; впаян стеклянный штифт, при $T\uparrow$ ртутьперемещает штифт, при $T\downarrow$ штифт остается на месте, ртуть уходит. Встряхнуть для нового использования\\
\textit{Минимальный:} Спиртовой. Штифтик из теммного стекла на концах уплотнения; при $T\uparrow$ спирт обтекает штиф: при $T\downarrow$ сдигается штиф вниз. Конец штифа удаленный от резервуара показывает мин T\\

\textbf{Термограф} - самопишущий прибор, устанавлиявается в спец (2) будке на метео площадке 2м над землей. Биметаллическая пластина изгибается при изм T и изменяет направление стрелки с пером.
