% !TEX root = ../main.tex

\section{Шаропилотный метод опр скорости и напр ветра по высотам, использование в ГА}
Шар-пилот - резиновая оболочка с водородом (легче воздуха) переносится ветром в гор плоскости\\
Наблюдая за перемещением (с помощью аэрологического (оптический или радио) теодолита) и зная его верт скорость можно опр-ить d и U гор перемещения по вертикальным и горизонтальным углам\\
Скорость шара зависит от плотности воздуха и шара\\
В течении первых 3 минут отсчеты производятся через 30с, затем через 1 минуту\\
$D=Hctg\delta$; D-гор удаление; H-высота; $\delta$ - верт. углы\\
$H=\omega t$; $\omega$- верт скорость\\
Далее используется аэрологический планшет для обработки данных (d,U термически ветер и эквивалентный ветер)\\

\textit{Различаются по диаметру} (10, 20, 30см); Оболочка №10 при слабом ветре и низкой облачности; №20 при облачности ср яруса и сильном ветре; №30 при малооблачной погоде\\
Имеют \textit{различный цвет:} светлые видны на синем фоне неба; черные на облачном.

