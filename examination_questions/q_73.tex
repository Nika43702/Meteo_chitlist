% !TEX root = ../main.tex

\section{Влияние физических параметров атм на полет ВС на эшелоне}
\textbf{При T↑ и P↓} плотность воздуха уменьшается и:
\begin{itemize}
	\item Падение тяги двигателя
	\item Уменьшение подъемной силы
	\item Уменьшение скороподъемности
	\item Повышение расхода топлива
	\item Потребная скорость горизонтального полета увеличивается
	\item Предельно допустимая высота полета уменьшается 
	\item Показания скорости и высотометра:  	Зимой -  завышены; Летом – занижены
	\item масса должна быть уменьшена
	\item меняется скорость звука ($a=20\sqrt{T}$)\\
\end{itemize}
\textbf{Температура:} T=0: обледенение полосы\\
T<0: обледенение\\

\textbf{Давление:} неравномерное расположение изобарических поверхностей приводит к изменению истинной высоты самолета\\

\textbf{Влажность:} наличие пара приводит к образованию явлений, ухудшающих видимость (туман, дымка), к образованию облаков, осадков, гроз.

