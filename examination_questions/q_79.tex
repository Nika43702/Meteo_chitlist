% !TEX root = ../main.tex

\section{Доп инфа}
\textbf{Ветер} - горизонтальное движение воздуха вдоль земной поверхности, характеризуемое скоростью и направлением. \\
Из-за неодн-ного расп-ия T и P вдоль земли возникает гор-ное движение воздуха из области высокого P в область низкого.\\
Структура ветра носит турбулентный характер, поэтому нарушаемое равновесие аэродин-их сил, за счет которых возникают добавочные ускорения, вызывают вредные перегрузки и болтанку\\
\textbf{Основные характеристики:}
\begin{itemize}
	\item Скорость.[м/с,км/ч,узлы] (1узел=2км/ч=0,5м/с)
	\item Направление - принята та чаcть горизонта, откуда дует ветер. (Измеряется в градусах/румбах(16шт))\\
	В навигации направление-куда дует ветер.
\end{itemize} 
\textbf{Классификация по скорости:}
\begin{itemize}
	\item слабый 2-3м/с
	\item умеренный 4-7м/с
	\item сильный 10-12м/с
	\item очень сильный >15м/с
	\item шторм >20м/с
	\item ураган >30м/с
\end{itemize}
\textbf{Типы:}
\begin{itemize}
	\item По направлению\\
	- \hspace{2ex} Постоянный\\
	- \hspace{2ex} Переменный: за 2 мин изменение более, чем на 1 румбу ($22,5\degree)$
	\item По скорости\\
	- \hspace{2ex} Ровный\\
	- \hspace{2ex} Порывистый: за 2 мин скорость изменяется на 4м/с и более
\end{itemize}
\textbf{Шквал} - кратковременное усиление ветра $\geqslant15$м/с со знач изменением направления\\

{\scriptsize \textbf{Прогноз туманов}}\\
\textbf{Радиационныt:} \textit{условия:} центр-ые части антициклонов, осей гребней, бар-ие седлавины, размытые бар поля\\
\textit{Способствуют:} устойчивая стратифи-ция атм-ы; часто с приземной инверсией и достаточная длительность периода охлаждения\\

\textbf{Фронтальные:} дается на основании прогноза перемещения фронта, прогноза ветра и фазового состояния осадков в зоне фронта\\

\textbf{Адвективный:} учитывается перемещение зон тумана, адвект-ые изменения темп и точки росы; возможности сниж-ия облаков до поверхности земли; охлаждение в процессе ночго радиацонного выхолаживания\\
Над отрк морем прогноз сводится к прогнозу смещения массы, распол-шейся над более теплой повер-ти, на отно-о холодну. часть моря\\
\textit{Способствуют:} большие градиенты темп-ы вдоль траектории перемещения воз. масс\\
\textit{Прогноз рассеяния:} основ-ся на учете факторов:
\begin{itemize}
	\item Прекращ-ия адвекции тепла в связи с измен-ем направленности ветра
	\item Прохождение теплого сектора циклона или окклюдирование циклона
	\item Понижение точки росы в связи с конденсацией пара
	\item Возрастание верт-ого турб-ого обмена при понижении удельной влж с высотой
	\item Выпадение интенсивных осадков, вызывающих процесс переконденсации (мелкие испар-ются, большие растут) и когуляции
\end{itemize}
\textbf{Туманы испарения:} перемещение холодной массы на открытую водную повер-ость. Также учит-ется верт градиент темп-ы (чем меньше, тем вероятнее туман) и скорость ветра (слабый для тумана)