% !TEX root = ../main.tex

\section{Инверсия темп, условия формирования и влияние на полет}
\textbf{Инверсия ($\gamma<0$)}– задерживающий слой, гасящий вертикальное движение воздуха, под которым происходит скопление водяного пара или других твердых частиц, ухудшающих видимость. В результате образуются туманы и различные виды облаков.\\
Во многих случаях инверсии – разрыв ветра (над и под инверсии). Резкое изменение скорости и направления ветра.\\
\textbf{Инверсии бывают:}
\begin{itemize}
	\item \textbf{Радиационные:} возникают вблизи земной поверхности, вследствие излучения ею большого количества тепла. (ночь, зима-в течении суток на некс сотен м)\\
	\item \textbf{Адвективные:} образуются при перемещении теплого воздуха по холодной поверхности. Нижний слой воздуха охлаждается от земли и передает холод выше путем турбулентного перемешивания.\\
	Возникают на высоте несколько сотен метров от зем поверхности\\
	Верт мощность: неск десятков метров
	\item \textbf{Инверсия сжатия/оседания}: образуются в области антициклона в результате опускания воздуха и адиабатического нагревания верхнего слоя (1/100м). Опускающийся нагретый воздух не опускается до приземных слоев из-за трения их о зем поверзность, а растекается образуюя слой инверсии\\
	Большая гор протяженность, ширина до неск сотен м, чаще всего на высоте от 1-3 км)
	\item \textbf{Фронтальные:} связаны с фронтальными разделами, где холодный воздух подтекает клином под теплую воздушную массу. Переходный слой между воздушными массами – фронтальная зона – слой инверсии, толщиной несколько сотен метров.
\end{itemize}
Под ними могут образовываться: дымка, туман, облака. Болтнака на волнах (границы инверсий). Под высотными инверсиями: усиление ветра и струйные течения (под тропопаузой)
