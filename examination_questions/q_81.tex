% !TEX root = ../main.tex

\section{Атмосферное давление2}
\par \textbf{Геопотенциальный метр} численно равен работе, затраченной на подъем 1т воздуха на 1м при g=9,8 $m/c^2$. Характеризует качественную сторону\\ 
\textbf{Линейный метр} характеризует количественную сторону. 
Гпм отличается от лин. метра на 0,3\%\\ 
$H_{\text{гпм}}=67,44T_{cp}lg\frac{P_{0}}{P_{H}}$	
\par \textbf{Изобара} - пересечение изобарических поверхностей с УМ \\
\textbf{Барическое поле:} картина давления, где бар поверхности наносятся на карту погоды каждые 5гПа, показывающая распределение давления воздуха\\
\textbf{Формы бар. поля:} 
\begin{itemize} 
	\item \textbf{Циклон (барический мин)} - область пониженного P, ограниченная замкнутыми изобарами с наименьшим P в центре.\\ 
	\textit{Ветер:} против ч.с. к центру
	\item \textbf{Антициклон (бар макс)} - область повышенного P, ограниченная замкнутыми изобарами с наибольшим P в центре.\\ 
	\textit{Ветер:} по ч.с. из центра 
	\item \textbf{Ложбина} - вытянутая область пониж. P, вдоль которой можно провести ось ложбины. (Не замкнутые изобар. поверхн) 
	\item \textbf{Гребень} - вытянутая область повышен. P, вдоль которой (по наибольшим p) можно провести ось. (Не замкнутые изобар. поверхн) 
	\item \textbf{Седловина} - бар. поле между двумя крест накрест лежащими двумя областями циклона и антициклона 
\end{itemize} 

\textbf{Условия погоды:} 
\begin{itemize} 
	\item[] Плохие: циклон, ложбина 
	\item[] Благоприятные: антициклон, гребень 
\end{itemize} 

\textbf{Эешелон} - выделяемое для полётов относительная барометрическая высота, отсчитываемая от изобарической поверхности 1013 гПа.(фактически полет по барометрическим поверхностям)

\par \textbf{Уравнения статики:}\\ 
Для однородной атм ($\rho=const$): $P_{H}=P_{0}-\rho gH$ \\ 
Для изотерм-ой атм ($T=const$): \\
$lnP_{H}-lnP_{0}=-\frac{1}{RT}gH$ \\

\textbf{Суточный ход:} в основном непериодичный\\ 
За день может меняться на 20-30 гПа\\ 
Суточная амплитуда в среднем 3-4гПа\\ 
В тропиках: 2 максимума и 2 минимума\\ 
К 60 широтам: уст амплитуда-десятые доли гПа\\ 
\textbf{Мин:} утром/после обеда \\
\textbf{Макс:} перед полуднем/полуночью \\
\textbf{Годовой ход:} 
\begin{itemize} 
\item \textit{Умеренные широты:} \\ 
Над океанами: циклоны глубже зимой, чем летом\\ 
Над материками: антициклоны-зимой, циклоны-летом 
\item \textit{Субтропики:} \\ 
Над океанами: антициклоны в Северном полушарии: в июле; в Южном: в январе
\item \textit{Экватор:} циклоны весь год 
\item \textit{Над Арктикой:} антициклоны, сам антициклон над Гренландией 
\item \textit{Над Антарктикой:} антициклон устойчивый весь год 
\end{itemize} 

\par \textbf{Типы годового хода:} 
\begin{itemize} 
\item Над материками: max-зимой; min-летом. Амплитуда увел с удавлением от океана 
\item \textit{Высокие широты:} max-летом; min-зимой 
\item \textit{Средние широты:} max-летом и зимой; min-весной и осенью 
\item \textit{Тропики:} годовой ход выражен слабо 
\end{itemize}

\par \textbf{Приборы для измерения давления} 
\begin{itemize} 
	\item \textbf{Станционный чашечный ртутный барометр}: Трубка с ртутью\\ 
	Поправки: инструм, темп, на высоту над ур моря, на место станции 
	\item \textbf{Металлический барометр-анероид}: деформация анероидой мембранной коробки\\ 
	Поправки: шкаловая, темп, добавочная 
	\item \textbf{Барограф:} для непрерывной регистрации давления
\end{itemize}
\vspace*{5\baselineskip}

