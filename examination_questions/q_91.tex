% !TEX root = ../main.tex

\section{Облака2}
\par \underline{\textbf{Классификация:}}\\
\par \textbf{I. Генетическая (по происхождению)}

\begin{itemize}
	\item \textbf{\textit{Кучевообразные:}} Различные виды конвекции\\
{
	\setlength{\columnseprule}{0pt}
\begin{multicols*}{2} 	Кучевые (Cu)\\
	Кучево-дождевые (Cb)\\
	Высококучевые (Ac)\\
	Перисто-кучевые (Cc)
\end{multicols*}
}
	\item \textbf{\textit{Волнистообразные:}} Появляются при волновых колебаниях на слоях инверсии, изотермии и слоях с небольшим вертикальным градиентом T\\
	Перисто-кучевые волн (Cc und – undulatus)\\
	Высококучевые волн (Ac und)\\
	Слоисто-кучевые (Sc)\\
	\item \textbf{\textit{Слоистообразные:}} Возникают в результате восходящего скольжения воздуха вдоль пологих фронтальных разделов  \\
	Перистые (Ci)\\
	Перисто-слоистые (Cs)\\
	Высокослоистые (As)\\
	Слоисто-дождевые (Ns)
\end{itemize}



\par\underline{\textbf{II. По микрофизическому строению:}}\\
В результате водяного пара образуются:

\begin{itemize}
\item \textit{Мелкие капли} (диаметр около 50мкм (микрон))
\item \textit{Ледяные кристаллы} (вид 6-и гранной призмы)
\end{itemize}

\par \textbf{Облака:}
\begin{itemize}
\item \textbf{Капельножидкие} (из капель): водность: 0,01-4 (гр/м3)\\
T>-12 в общем случае, но до -40 капли находятся в переохлажденном состоянии
\item \textbf{Кристаллические/ледяные:} водность: <0,02 (гр/м3)\\
T<-40
\item \textbf{Смешанные:} водность: ≤2-3 (гр/м3)\\
\end{itemize}


\par \underline{\textbf{III. Морфологическая (форма):}}
\begin{itemize}
	\item \textit{\textbf{Верхний Ярус:}} 6км – до тропопаузы; кристаллы\\
	Ci Перистые \\
	Cc Перисто-кучевые \\
	Cs Перисто-слоистые \\
	Опасны метловидные вершины над Cb; облачные полосы, связанные с сильными ветровыми потоками струйных течений\\
	Движение  с Запада Ci и Cs – признак ухудшения погоды и выпадения осадков.\\
	Наличие плотных Ci при высокой T и d – неустойчивость атмосферы и возможность гроз (теплый фронт)
	
	\item \textbf{\textit{Средний Ярус:}} 2 -6 км; Кристаллы и смешанные\\
	Ac Высококучевые\\
	As Высокослоистые\\
	Башенкообразные и хлопьевидные As – неустойчивость состояний атм-ы и болтанка; предвещают развитие гроз;
	Высококучевые чечевицеобразные (Lenticularis) предвещают cold фронт; в горах – предветренные стоячие волны
	Уплотнение As – приближение  warm фронта с обложными осадками; возможно обледенение в As\\
	
	\item \textbf{\textit{Высокий:}} Ниже 2 км; Капли Зимой: переохлажденные капли и кристаллы со снежинками\\
	Sc Слоисто-кучевые\\
	St  Слоистые\\
	Ns Слоисто-дождевые\\
	Cu Кучевые (кроме мощных)\\
	Ns вертикальная мощность (неск км) размытая н.г. из-за осадков\\
	Зимой: ВНГО ниже 100м. Под Ns – дождевые облака (Cu Fr) {\hspace{1cm} В Ns T<0 обледенение}\\
	Летом: Ns выше, дождевые облака реже.\\
	В Ns грозовые очаги, чаще ночью\\
	St и Sc изменчивость ВНГО (<300м)\\
	Sc часто внутримассовые особенно зимой, испытывают волновые движения
	
	\item \textbf{\textit{Вертикального развития:}} НГО-ниж. яр; ВГО-ср/верх яр\\
	 Cu Congestus Кучевые мощные\\
	 Cb Кучево-дождевые\\
	 В области Cb: сильная болтанка, верт броски на сотни метров, сильное обледенение. Особенно опасна передняя часть Cb, под основанием «крутящий вал» с горизонтальной осью (шквальный ворот)
\end{itemize}