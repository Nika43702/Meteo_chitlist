% !TEX root = main.tex

%%% For testing document:
%\usepackage{showframe}
\usepackage{blindtext}
\usepackage{lipsum}
\usepackage{comment}
%

% Картинки
\usepackage{wrapfig}
\usepackage{array}
\usepackage[dvipdfm]{graphicx}
\graphicspath{{Pictures/}}
\DeclareGraphicsExtensions{.pdf,.png,.jpg}

%%% Работа с русским языком
\usepackage{cmap}                   % Поиск в PDF
\usepackage{mathtext}               % Русские буквы в фомулах
\usepackage[T2A]{fontenc}           % Output font encoding for international characters
\usepackage[utf8]{inputenc}        % Required for inputting international characters
\usepackage[english,russian]{babel} % Локализация и переносы
\usepackage{indentfirst}            % Русский стиль: отступ первого абзаца раздела
\usepackage{textcomp} 				%символы в тексте

\frenchspacing                      % Русский стиль: Промежутки между словами и между    
% предложениями отличаться не должны. 
%

%%% Отступы:
\usepackage[nomarginpar, top=10mm,bottom=10mm,left=7.5mm,right=8.5mm , headheight=0pt, ]{geometry}%headsep=0pt
%

%%% Дополнительная работа с математикой
\usepackage{amsmath, amsfonts, amssymb, amsthm, mathtools} % AMS
\usepackage{icomma} % "Умная" запятая: $0,2$ --- число, $0, 2$ --- перечисление
%

\usepackage{lastpage}
\usepackage[nodisplayskipstretch]{setspace}  % "Умное" расстояние между строк 

\usepackage{misccorr} % в заголовках появляется точка, но при ссылке на них ее нет
\usepackage{indentfirst} % после заголовков ставится абзацный отступ


%%%Colour
\usepackage{soul} % Highlight text
\usepackage[dvipsnames]{xcolor}

\definecolor{Bleudefrance}{rgb}{0.19, 0.55, 0.91}
\definecolor{Brightgreen}{rgb}{0.4, 1.0, 0.0}
\definecolor{OliveGreen}{cmyk}{0.64,0,0.95,0.40}

%\sethlcolor{Brightgreen}
%\everymath\expandafter{\the\everymath \color{Bleudefrance}}
%\everydisplay\expandafter{\the\everydisplay \color{Bleudefrance}}
%

\usepackage{titling}         % Customizing the title section
\usepackage{hyperref}        % For hyperlinks in the PDF

\usepackage{mathdesign}
\usepackage{multicol}

%/////////////////////

\usepackage{ gensymb } % Цельсий
\usepackage{ upgreek } % Uptau

\usepackage[framemethod=tikz]{mdframed}
\usepackage{microtype}

%\renewcommand{\baselinestretch}{.8}
\pagestyle{empty}           % Turn off header and footer

\setlength{\parindent}{0pt}
\setlength{\parskip}{0pt plus 0.5ex}
%\setlength{\parskip}{1ex plus 0.5ex minus 0.2ex}

\usepackage{enumitem}
\setlist{leftmargin=3mm} % Отступ слева у item
%\setlist{nolistep, itemstep=0.3cm, parser=0pt}


%%% Redefine section commands to use less space
\makeatletter
\renewcommand{\section}{\@startsection{section}{1}{0mm}%
	{.2ex}%
	{.2ex}%
	{\sffamily\scriptsize\bfseries}}


\renewcommand{\subsection}{\@startsection{subsection}{1}{0mm}%
	{.2ex}%
	{.2ex}%
	{\sffamily\bfseries}}

\renewcommand{\subsubsection}{\@startsection{subsubsection}{1}{0mm}%
	{.2ex}%
	{.2ex}%
	{\sffamily\bfseries}}

\renewcommand{\paragraph}{\@startsection{paragraph}{1}{0mm}%
	{.2ex}%
	{.2ex}%
	{\sffamily\bfseries}}

\renewcommand{\subparagraph}{\@startsection{subparagraph}{1}{0mm}%
	{.2ex}%
	{.2ex}%
	{\sffamily\bfseries}}

\makeatother
\setlength{\parindent}{0pt}

%%%%%%%%%%%%%%%%%%%%%%%%%%%%%%%%
\makeatletter
\renewcommand{\@listI}{%
	\topsep=0pt }
\makeatother

\makeatletter
\let\old@itemize=\itemize
\def\itemize{\old@itemize
	\setlength{\itemsep}{0pt}
	\setlength{\parskip}{0pt}
	\setlength{\leftskip}{0pt}
}
\makeatother

\makeatletter
\let\old@enumerate=\enumerate
\def\enumerate{\old@enumerate
	\setlength{\itemsep}{0pt}
	\setlength{\parskip}{0pt}
	\setlength{\leftskip}{0pt}
}\makeatother

%%%%%%%%%%%%%%%%%%%%%%%%%%%%%%%

%%% Сужение списков
\newenvironment{itemize*}%
{\begin{itemize}%
		\setlength{\itemsep}{1pt}%
		\setlength{\parskip}{1pt}}%
	{\end{itemize}}

\newenvironment{enumerate*}%
{\begin{enumerate}%
		\setlength{\itemsep}{1pt}%
		\setlength{\parskip}{1pt}}%
	{\end{enumerate}}
%

% Don't print section numbers
%\setcounter{secnumdepth}{0}

\righthyphenmin=2                       % Минимальное число символов при переносе - 2
